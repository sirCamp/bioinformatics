% !TEX encoding = UTF-8
% !TEX TS-program = pdflatex
% !TEX root = ../tesi.tex
% !TEX spellcheck = it-IT

%**************************************************************
\chapter{Source Code}
\label{cap:source code}
%**************************************************************

\intro{In this chapter is reported the source code of the algorithms developed by me and used inside the projects}\\

%**************************************************************
\section{Phyton source code}

In this section is reported all the pyhton source codes use in the project
%%%%%%%%%%%%%%%%%%%%%%%%%%%%%%%%%%%%%%%%%%%%%%%%%%%%%%%%%%%%%%%%%%%%5
\subsection{Insertion length}
\tiny
\begin{minted}{python}

class FirstAlgorithm(BaseAlgorithm):

    def __init__(self, name, path):
        BaseAlgorithm.__init__(self, name, path)

    def run(self):
        self.lengthOfCoverage()
        genome_insertion = []
        sum = 0;
        sumI = 0;

        ext = 0;
        with open(self.path) as file:
            for line in file:

                if ext == self.length:break;

                if line.find("@") != -1:
                    continue

                pieces = line.split("\t")
                ext += 1
                if int(pieces[8]) > 0 and int(pieces[8]) < 10000:
                    isize = abs(int(pieces[3])-int(pieces[7]))
                    genome_insertion.append(isize)
                    sum += isize
                    sumI += 1


        file = "position,isize\n"
        scatFile = "position,distance\n"

        isizeM = int((sum / sumI))
        tmp = 0
        index = 1
        for i in genome_insertion:
            file += str(index)+","+str(i)+"\n"
            dist = (i-isizeM)
            scatFile += str(dist)+","+str(i)+"\n"
            tmp += math.pow(dist,2) #per la dev standard
            index += 1


        print "Saving genome_insert_csv.csv file..."
        targetF = open("genome_insert_csv.csv","w")
        targetF.truncate()
        targetF.write(file)
        targetF.close()
        print "Saving genome_insert_scat_csv.csv file..."
        targetF = open("genome_insert_scat_csv.csv","w")
        targetF.truncate()
        targetF.write(scatFile)
        targetF.close()
        print "Media: "+str(isizeM)
        print "Standard Dev: "+str(math.sqrt(tmp/(index)))
        print("Done!")

\end{minted}

%%%%%%%%%%%%%%%%%%%%%%%%%%%%%%%%%%%%%%%%%%%%%%%%%%%%%%%%%%%%%%%%%%%%%%%%%5
\subsection{Physical Coverage}
\tiny
\begin{minted}{python}
class PhysicalCoverageAlgorithm(BaseAlgorithm):

    leght = 0;

    def __init__(self, name, path):
        BaseAlgorithm.__init__(self, name, path)

    def run(self):

        self.lengthOfCoverage()
        genome_change = [0] * self.length

        for i in range:
            genome_change.append(0)

        with open(self.path) as file:
            for line in file:
                if line.find("@") != -1:
                    continue;

                pieces = line.split("\t")

                if ((eval(pieces[1]) & 3) == 3) and (int(pieces[8]) > 0):
                    genome_change[int(pieces[3])] = genome_change[int(pieces[3])] + 1

                    genome_change[int(pieces[7]) + len(str(pieces[9]))] = (genome_change[int(pieces[7]) + len(str(pieces[9]))]) - 1

        file = "";
        file += "fixedStep chrom=genome start=1 step=1 span=1\n"
        currentC = 0



        for i in range:
            currentC += genome_change[i];
            file += str(currentC)+"\n"

        print "Saving to physical_coverage.wig "
        targetF = open("physical_coverage.wig","w")
        targetF.truncate()
        targetF.write(file)
        targetF.close()
        print("Done!")
\end{minted}


%%%%%%%%%%%%%%%%%%%%%%%%%%%%%%%%%%%%%%%%%%%%%%%%%%%%%%%%%%%%%%%%%%%%%%%%%%%%
\subsection{Sequence Coverage}
\tiny
\begin{minted}{python}

class SequenceCoverageAlgorithm(BaseAlgorithm):



    def __init__(self, name, path):
        BaseAlgorithm.__init__(self, name, path)

    def run(self):

        self.lengthOfCoverage()
        genome_change = [0] * self.length

        #open the SAM file
        print "Analising SAM file..."
        with open(self.path) as file:
            for line in file:

                #exclude header lines
                if line.find("@") != -1:
                    continue;

                pieces = line.split("\t")

                if ((eval(pieces[1]) & 3) == 3) and (int(pieces[8]) > 0):
                   index = 0

                   for x in xrange(0,len(str(pieces[9]))):
                       genome_change[int(pieces[3])+index] = genome_change[int(pieces[3])+index] + 1
                       index +=1
                else:
                    index = 0
                    if int(pieces[8]) < 0:
                      for x in xrange(0,len(str(pieces[9]))):
                        genome_change[int(pieces[3])-index] = genome_change[int(pieces[3])-index] + 1
                        index +=1



        file = "";
        file += "fixedStep chrom=genome start=1 step=1 span=1\n"

        coverage = 0

        for i in genome_change:
            coverage = genome_change[i];
            file += str(coverage)+"\n"

        print "Saving sequence_coverage.wig file..."
        targetF = open("sequence_coverage.wig","w")
        targetF.truncate()
        targetF.write(file)
        targetF.close()
        print("Done!")
\end{minted}


%%%%%%%%%%%%%%%%%%%%%%%%%%%%%%%%%%%%%%%%%%%%%%%%%%%%%%%%%%%%%%%%%%%%%%%%%%%%
\subsection{Kmers Counting}
\tiny
\begin{minted}{python}

class KmersAlgorithm(BaseAlgorithm):
	
	def __init__(self, name, path):
    		BaseAlgorithm.__init__(self, name, path)

    def mapping(self, string):

        string = string.replace("A", "W")
        string = string.replace("T", "A")
        string = string.replace("W", "T")

        string = string.replace("C", "W")
        string = string.replace("G", "C")
        string = string.replace("W", "G")
        return string

    def run(self):

        self.lengthOfCoverage()

        #open the SAM file
        print "Open SAM file..."
        kmers = {}
        index = 0
        print "Looking for kmers and it's opposites for "+str(self.length)
        with open(self.path) as file:
            for line in file:

                if index == self.length:
                    break;

                index += 1
                #exclude header lines
                if line.find("@") != -1:
                    continue;

                pieces = line.split("\t")

                read = str(pieces[9])
                padding = 0
                while len(read[padding:padding+4]) >= 4:
                    key = read[padding:padding+4]
                    padding += 1
                    if kmers.has_key(key):
                        kmers[key] = kmers[key] +1

                    elif kmers.has_key(self.mapping(key)):
                            kmers[self.mapping(key)] = kmers[self.mapping(key)] +1
                    else:
                        kmers.__setitem__(key,1)

        file = "";
        file = "kmer,quantity\n"
        for i in kmers:
            file += str(i)+","+str(kmers[i])+"\n"

        print "Saving kmers.csv file..."
        targetF = open("kmers.csv","w")
        targetF.truncate()
        targetF.write(file)
        targetF.close()
        print("Done!")
\end{minted}

\section{R source code}
In this section is reported all the R source codes use in the project


\subsection{Insertion length distribution}
\tiny
\begin{minted}{matlab}
chart <- read.csv('../genome_insert_csv.csv', header=TRUE, sep=",")
png('../genome_insert.png')
barplot(
	as.integer(chart$isize), 
	xlim=c(0,20000),
	ylim=c(0,28000) , 
	width = 5,
	main="Insertion length distribution", 
	border = "dark blue", 
	xlab="Isertion length", 
	ylab="Isertions", 
	names.arg = chart$position, 
	horiz = TRUE, 
	col="grey"
)
\end{minted}


\subsection{Kmers distribution}
\tiny
\begin{minted}{matlab}
chart <- read.csv('../kmers.csv', header=TRUE, sep=",")
png('../kmers.png',width = 2000,  height = 600)
barplot(
	as.integer(chart$quantity), 
	xlim=c(0,900), ylim=c(0,15000) , 
	width = 5, 
	main="Kmers distribution", 
	border = "dark blue", 
	xlab="Kmers", 
	ylab="Quantity", 
	names.arg = chart$kmer, 
	horiz = FALSE, 
	col="grey", 
	las=2
)
\end{minted}