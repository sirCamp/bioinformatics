% !TEX encoding = UTF-8
% !TEX TS-program = pdflatex
% !TEX root = ../tesi.tex
% !TEX spellcheck = it-IT

%**************************************************************
\chapter{Results and Considerations}
\label{cap:considerations}
%**************************************************************

\intro{This chapter is about the final results and considerations about the project}\\
\section{Results}
\normalsize
The results of the mean and standard deviation of genome insertion length are:
\begin{itemize}
\item \begin{description}
		\item[MEAN:] 2102
  \end{description}
\end{itemize}

\begin{itemize}
\item \begin{description}
		\item[STD:] 236.85
  \end{description}
\end{itemize}

The obtained results are some structural variation found in the resequenced genome.
These structural variation are:
\begin{itemize}

\item \begin{description}
		\item[Insertion] a insertion is visible at 1.050k bases, this insertion is confirmed by the sequence coverage
  \end{description}
\item \begin{description}
		\item[Long deletion] a long deletion is visible between 1.380k bases and 1.400k bases, this deletetation is confirmed by the sequence coverage
  \end{description}
\end{itemize}
\begin{itemize}
\item \begin{description}
		\item[Inversion] the inversion structural modification is visible between 1.745k bases and 1.770k bases
  \end{description}
\end{itemize}
\begin{itemize}
\item \begin{description}
		\item[Short deletion] a long deletion is visible at 2.448k bases, this deletetation is confirmed by the physical coverage
  \end{description}
\end{itemize}



As I wrote in the section \ref{sec:cigar}, the cigar's track allows and helps to find these structural variations.\\\\

About the kmers, the results are really varius and with larger \emph{k} size the number of kmer quantity decreases but the number of kmers becomes really huge.
We can see that for the \emph{k = 4}, the most frequent kmers are: 
\begin{itemize}
\item TTTT
\item AAAC
\item GTTT
\end{itemize}

For the \emph{k = 7}, the most frequent kmers is: 

\begin{itemize}
\item TTGAAAA
\end{itemize}

At last, for the  \emph{k = 9}, the most frequent kmers is:
\begin{itemize}
\item AAAACTTTT
\end{itemize}
 
\section{Considerations}

Some difficulties can be encountered in this project:
\begin{itemize}
\item \begin{description}
		\item[Information] the information retrieval is really difficult, the sources available  are often poor and always confused, for the same thing it's really easy to find different definitions and different information. The most used blog and websites are \href{https://www.biostars.org/}{Biostars} and \href{http://seqanswers.com/forums/}{SEQ answer Forum}, but often their information is not correct,  \href{https://it.wikipedia.org}{WeikiPedia} is used too, but often it's not clear about these topics.\\ I suppose these problems are caused by the young age of Bioinformatics.
  \end{description}
\end{itemize}


\begin{itemize}
\item \begin{description}
		\item[The approach] to approach the project is not simple, there are many languages that could be used, but each one hides a problem.\\
		It's necessary to evaluate each one of these programming languages by these:
		\begin{itemize}
			\item semplicity
			\item speed
			\item strin manipulation
			\item used RAM
			\item used CPU
		\end{itemize}
  \end{description}
\end{itemize}


\begin{itemize}
\item \begin{description}
		\item[Hardware] The hardware of the computer must be really powerful, otherwise executing some tasks is not simple
  \end{description}
\end{itemize}

\section{Conclusions}
The project has allowed me to search and test this new field of computer science.
The Bioinformatics will be the future of the computer science because it's very useful in finding and identifying some deseases, genomic variations, and this implicitly explains how it helps us prevent the health problem for humans as well as for animals and plants.\\\\
The Bioinformatics could be helpful in each area of the Biology.